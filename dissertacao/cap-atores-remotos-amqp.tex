\chapter{Atores remotos com o padrão AMQP}
\index{Atores remotos com AMQP}
\label{cap:atores-remotos-com-amqp}
