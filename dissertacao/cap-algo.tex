\section{Entidades conectadas ponto-a-ponto via \english{sockets}}
\index{Estrutura!ponto-a-ponto}
\label{sec:entidades-ponto-a-ponto}

Em uma transferência de mensagens entre duas entidades conectadas 
ponto-a-ponto por \english{sockets} TCP ou UDP, ambas as entidades precisão saber
detalhes de conexão e do protocolo das mensagens a serem enviadas. 
O fato de não haver uma entidade intermediando a troca
de mensagens, permite que uma das entidades identifique uma eventual desconexão
da outra, ou ainda que saiba o endereço do nó onde está a outra entidade.
Em casos onde as entidades estejam em nós que não fazem parte de uma rede interna,
a informação de localidade pode não indicar exatamente o nó corrente da entidade, 
mas o endereço de alguma porta de ligação (\english{gateway}). 

Uma característica importante na transferências de mensagens entre entidades 
conectadas ponto-a-ponto via \english{sockets}, é que cada \english{socket} utiliza 
exclusivamente uma porta para aceitar as conexões remotas, criando uma relação de um 
para um entre portas e entidades. 

Os protocolos TCP e UDP utilizam para a numeração
de portas em seus cabeçalhos inteiros de $16$ bits sem sinal, limitando o número de
portas a $65536$ ($0$ -- $65535$)\cite{tcp}. Ademais, algumas portas que são utilizadas 
para serviços comuns como, por exemplo servidores de correio eletrônico, 
possuem numeração fixa definida pela IANA (\english{Internet Assigned Numbers Authority})
e não podem ser abertas para uso geral. 

%% todo.. falar com reverbel para pegar paper : http://portal.acm.org/citation.cfm?id=1101798&dl=ACM&coll=DL&CFID=29058066&CFTOKEN=41896717

%% argumentar sobre questoes básicas de segurança e firewall ?

Este tipo de abortagem não implica no uso de armazenamento intermediário das mensagens. Uma
mensagem enviada de uma entidade para outra, até poderia ser armazenada pela entidade que 
recebe a mensagen, mas não pelo mecanismo de transporte. Fica a cargo da entidade que recebe 
a mensagem implementar o armazenamento temporário caso haja essa necessidade.

\subsection{Atores remotos conectados ponto-a-ponto}
\label{subsec:atores-remotos-ponto-a-ponto}

Uma implementação de atores remotos que não imponha como limite à quantidade
de atores remotos a serem criados em um nó o número de portas disponíveis no
próprio nó deve, de alguma maneira permitir que conjuntos de atores sejam
associados às portas.

Na implementação de atores remotos do Akka apresentada na seção 
\ref{sec:atores_remotos_akka}, vimos que os atores são agrupados por
\english{host} e porta, sendo armazenados no \lstinline$RemoteServerModule$. 
Vimos também na seção \ref{sec:atores_locais} que atores são identificáveis, 
o que permite que eles sejam localizados no \lstinline$RemoteServerModule$ 
para que possam fazer o recebimento e processamento das mensagens. 

%% investigar mais a fundo o jboss netty e comentar de alguns detalhes dele aqui.
Ainda que a implementação de atores remotos do Akka agrupe
os atores, removendo a relação do limite do número de atores com o 
número de portas disponíveis, as entidades \lstinline$RemoteServerModule$ 
e \lstinline$RemoteClientModule$ ainda assim possuem o conhecimento 
do \english{host} e porta aonde se está conectado e não fazem armazenamento
intermediário das mensagens.

%% entrar mais em detalhes sobre erlang?
A implementação de atores remotos de Erlang não usa explicitamente portas. 
Cada máquina virtual Erlang possui um nome associado, e esse nome é utilizado
junto com a informação do \english{host} durante a criação de atores remotos.
O nome da máquina virtual é definido durante a inicialização da máquina virtual
via parâmetro \lstinline$-name$. Diversas máquinas virtuais podem estar em
execução em um determinado \english{host} e são unicamente identificadas
por \lstinline$name@host$. Vale ressaltar que as máquinas virtuais Erlang
que estão em uma mesma rede de computadores estão por padrão em \english{cluster}.
