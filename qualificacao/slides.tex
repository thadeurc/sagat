\documentclass{beamer} 
\usepackage{beamerthemeshadow}
\usepackage[latin1]{inputenc}
\usepackage[brazil]{babel}

\title[AMQP para transportar Mensagens entre Atores Remotos]{Troca de Mensagens entre Atores Remotos} 
\author{Thadeu de Russo e Carmo \\ \texttt{thadeurc@ime.usp.br}} 
\institute{Orientador: Prof. Dr. Francisco da Rocha Reverbel\\ Instituto de Matem\'atica e Estat\'istica \\ Universidade de S\~ao Paulo}
\date{Novembro, 2010}

\begin{document}
\frame{\titlepage}
\section{Introdu\c{c}\~ao} 
\frame{
\frametitle{Motiva\c{c}\~ao}
	\only<1>{
		Necessidade de programas escritos de maneira concorrente:
		\begin{itemize}
			\item Fim da lei de Moore -- programas precisam ser escritos de maneira concorrente para poder
			usufruir dos ganhos de desempenho dos processadores com m\'ultiplos n\'ucleos
			
			\item Abordagem convencional com o uso de travas al\'em de complexa, possui
			limita\c{c}\~oes em rela\c{c}\~ao a componibilidade
		\end{itemize}
	}
	\only<2>{
		Alternativas ao uso de travas:
		\begin{itemize}
			\item \textit{Software Transactional Memory}(STM) -- mecanismo de controle an\'alogo
			as transa\c{c}\~oes de SGBDs	
			\item Atores -- troca de mensagem ass\'incrona entre processos
		\end{itemize}
	}
	\only<3>{
		\begin{beamerboxesrounded}{Defini\c{c}\~ao de um Ator}
			Agentes computacionais que especificam um endere\c{c}o para sua caixa de correio e seu comportamento,
			fazendo o processamento ass\'icrono das mensagens que recebe em sua caixa de correio. Mensagens recebidas
			s\~ao mapeadas em uma $3$-tupla consistindo de: 
			\begin{itemize}
				\item Um conjunto finito de mensagens enviadas para outros atores onde o endere\c{c}o \'e conhecido
				(incluindo o pr\'oprio ator)
				\item Um novo comportamento a ser usado para processar a mensagem seguinte
				\item Um conjunto finito de novos atores
			\end{itemize}
		\end{beamerboxesrounded}
	}
	\only<4>{
		\textit{Middlewares} orientados a mensagem (MOMs) possuem como uma de suas caracter\'isticas a troca 
		ass\'incrona de mensagens:
		\begin{itemize}
			\item Formam base para simplificar o desenvolvimento de sistemas
			\item Possuem suporte e mecanismos para gerenciamento de erros e garantia de entrega de mensagens	
			\item S\~ao frequentemente apresentados como tecnologia que pode mudar a maneira como sistemas distribu\'idos
			s\~ao constru\'idos
		\end{itemize}
	}
	\only<5>{
		Sinergia entre o Modelo de Atores e MOMs:
		\begin{itemize}
			\item Atores em diferentes n\'os podem trocar mensagens entre si
			\item MOMs prov\^em robustes para a troca de mensagens entre entidades em diferentes n\'os de um sistema
		\end{itemize}
	}
}
\frame{\frametitle{Objetivos}
	\begin{itemize}
		\item Estudar o modelo de atores e a implementa\c{c}\~ao feita no projeto Akka
		\item Fazer uma implementa\c{c}\~ao em Scala com base na implementa\c{c}\~ao feita pelo projeto Akka, 
		substituindo o uso de \textit{sockets} para o tr\'afego de mensagens pelo padr\~ao aberto de \textit{middleware}
		orientado a mensagem \textit{Advanced Message Queue Protocol} (AMQP)
		\item Uma an\'alise comparativa entre as duas abordagens
	\end{itemize}
}

\section{Trabalhos Relacionados}
\subsection{Atores em Erlang}
\frame{
\frametitle{Atores em Erlang}
	\begin{beamerboxesrounded}{Principais caracter\'isticas}
		\begin{itemize}
			\item Atores s\~ao processos ultra leves criados dentro da m\'aquina virtual Erlang
			\item Primitivas para cria\c{c}\~ao de novos atores (\textit{spawn}), envio de mensagens (!), 
			recebimento (\textit{receive}) e cria\c{c}\~ao de hieraquia de supervis\~ao (\textit{link}) est\~ao
			embutidos na linguagem
			\item \textit{Spawns} remotos s\~ao poss\'iveis e o envio das mensagens \'e feito com o uso
			de \textit{sockets} TCP e UDP
			\item Mensagens que n\~ao est\~ao definidas no comportamento do ator s\~ao recolocadas na 
			caixa de correio do ator para tentativa de processamento futuro
		\end{itemize}
	\end{beamerboxesrounded}
}
\subsection{Atores em Scala}
\frame{
\frametitle{Atores em Scala}
	\only<1>{A implementa\c{c}\~ao da biblioteca padr\~ao de Scala foi baseada na implementa\c{c}\~ao feita em Erlang, possuindo
	basicamente as mesmas funcionalidades j\'a descritas.
	}
	\only<2>{
		\begin{beamerboxesrounded}{Caracter\'isticas adicionais}
			\begin{itemize}
				\item Atores s\~ao baseados em \textit{threads} Java
				\item Possui m\'etodos como \textit{act} e \textit{react} a serem usados como alternativa ao \textit{receive}, 
				que permitem um reuso de \textit{threads} inativas
				\item Cria\c{c}\~ao de atores remotos (\textit{spawn remotes}) n\~ao s\~ao poss\'iveis
				\item Atores s\~ao acess\'iveis remotamente de modo an\'alogo ao uso de JNDI
			\end{itemize}
		\end{beamerboxesrounded}
	}
	\only<3>{
		\begin{beamerboxesrounded}{Outros tipos de envio}
			\begin{itemize}
				\item !? : faz o envio da mensagem e fica bloqueado dentro de um tempo limite aguardando a resposta
				\item !! : faz o envio da mensagem e recebe um resultado substitudo (\textit{Future}). Uma fun\c{c}\~ao parcial
				fica aguardando o resultado real
			\end{itemize}
		\end{beamerboxesrounded}
		\textit{Future} \'e um mecanismo usado para  retornar valores de chamadas ass\'incronas, e age como um substituto
		para o resultado que ainda n\~ao foi materializado.
	}	
}
\section{Advanced Message Queue Protocol}
\frame{
\frametitle{Advanced Message Queue Protocol}
	\only<1>{
		\begin{itemize}
			\item Protocolo aberto desenvolvido por um conjunto de empresas com o objetivo de padroniza\c{c}\~ao
			e redu\c{c}\~ao de custos na integra\c{c}\~ao de sistemas
			\item Al\'em da defini\c{c}\~ao do protocolo, define tamb\'em a sem\^antica dos servidos da aplica\c{c}\~ao servidora 
			\item Objetiva que as capacidades de MOMs estejam dispon\'iveis pervasivamente nas empresas
		\end{itemize}
	}
	\only<2>{
		\begin{beamerboxesrounded}{Principais componentes}
			\begin{itemize}
				\item Servidor, tamb\'em conhecido como \textit{broker}
				\item Filas
				\item \textit{Exchanges}
				\item \textit{Bindings}
				\item \textit{Virtual hosts}
			\end{itemize}
		\end{beamerboxesrounded}
		Sistemas AMQP o recebimendo e roteamento das mensagens s\~ ao separados do seu armazemento
		e encaminhamento. Em sistemas pr\'e-AMQP, estas tarefas s\~ao feitas por blocos monol\'iticos.
	}
	\only<3>{
		\begin{beamerboxesrounded}{Analogia com sistemas de correio eletr\^onico}
			\begin{itemize}
				\item Mensagens AMQP s\~ao an\'alogas a mensagens de correio eletr\^onico
				\item Filas s\~ao an\'alogas a caixas de correio
				\item \textit{Exchanges} s\~ao an\'alogas a agentes de transfer\^encia de mensagens (MTA), e
				com base nas chaves de roteamento (To, Cc e Bcc no caso de correio eletr\^onico), verifica
				as tabelas de registro e decide como enviar as mensagens para uma ou mais caixas de correio
				\item \textit{Bindings} correspondem a entradas nas tabelas de roteamento do MTA		
			\end{itemize}
		\end{beamerboxesrounded}
	}
}

\section{Atores Akka}
\frame{
\frametitle{Envio Local de Mensagens}

}
\frame{
\frametitle{Envio Remoto de Mensagens}

}
\section{Proposta} 
\frame{
\frametitle{Proposta}
alalaa
}

\subsection{Atividades}
\frame{
\frametitle{Cronograma}
crono
}

\end{document}
